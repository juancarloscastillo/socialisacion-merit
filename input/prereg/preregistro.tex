% Options for packages loaded elsewhere
\PassOptionsToPackage{unicode}{hyperref}
\PassOptionsToPackage{hyphens}{url}
\PassOptionsToPackage{dvipsnames,svgnames*,x11names*}{xcolor}
%
\documentclass[
  12pt,
]{article}
\usepackage{lmodern}
\usepackage{setspace}
\usepackage{amssymb,amsmath}
\usepackage{ifxetex,ifluatex}
\ifnum 0\ifxetex 1\fi\ifluatex 1\fi=0 % if pdftex
  \usepackage[T1]{fontenc}
  \usepackage[utf8]{inputenc}
  \usepackage{textcomp} % provide euro and other symbols
\else % if luatex or xetex
  \usepackage{unicode-math}
  \defaultfontfeatures{Scale=MatchLowercase}
  \defaultfontfeatures[\rmfamily]{Ligatures=TeX,Scale=1}
\fi
% Use upquote if available, for straight quotes in verbatim environments
\IfFileExists{upquote.sty}{\usepackage{upquote}}{}
\IfFileExists{microtype.sty}{% use microtype if available
  \usepackage[]{microtype}
  \UseMicrotypeSet[protrusion]{basicmath} % disable protrusion for tt fonts
}{}
\makeatletter
\@ifundefined{KOMAClassName}{% if non-KOMA class
  \IfFileExists{parskip.sty}{%
    \usepackage{parskip}
  }{% else
    \setlength{\parindent}{0pt}
    \setlength{\parskip}{6pt plus 2pt minus 1pt}}
}{% if KOMA class
  \KOMAoptions{parskip=half}}
\makeatother
\usepackage{xcolor}
\IfFileExists{xurl.sty}{\usepackage{xurl}}{} % add URL line breaks if available
\IfFileExists{bookmark.sty}{\usepackage{bookmark}}{\usepackage{hyperref}}
\hypersetup{
  pdftitle={La socialización de la meritocracia em el contexto familiar y escolar},
  colorlinks=true,
  linkcolor=blue,
  filecolor=Maroon,
  citecolor=Blue,
  urlcolor=Blue,
  pdfcreator={LaTeX via pandoc}}
\urlstyle{same} % disable monospaced font for URLs
\usepackage[margin=0.78in]{geometry}
\setlength{\emergencystretch}{3em} % prevent overfull lines
\providecommand{\tightlist}{%
  \setlength{\itemsep}{0pt}\setlength{\parskip}{0pt}}
\setcounter{secnumdepth}{-\maxdimen} % remove section numbering
\usepackage{caption}
\captionsetup[figure, table]{labelfont={bf},labelformat={default},labelsep=period}
\usepackage{graphicx}
\usepackage{float}
\usepackage{booktabs}
\usepackage{longtable}
\usepackage{array}
\usepackage{multirow}
\usepackage{wrapfig}
\usepackage{float}
\usepackage{colortbl}
\usepackage{pdflscape}
\usepackage{tabu}
\usepackage{threeparttable}

\title{La socialización de la meritocracia em el contexto familiar y escolar}
\usepackage{etoolbox}
\makeatletter
\providecommand{\subtitle}[1]{% add subtitle to \maketitle
  \apptocmd{\@title}{\par {\large #1 \par}}{}{}
}
\makeatother
\subtitle{Pre-registro}
\author{}
\date{\vspace{-2.5em}2021-03-19}

\begin{document}
\maketitle

\setstretch{1}
Para la elaboración de este pre-registro se ha seguido la plantilla de
preinscripción de \href{https://aspredicted.org/}{AsPredicted.org}.

\hypertarget{resumen}{%
\subsection{Resumen}\label{resumen}}

Estudios en el área de la educación cívica y formación ciudadana han
generado valiosa evidencia sobre la socialización política y la
preparación para la ciudadanía democrática. Desde la socialización
política se ha estudiado la transmisión intergeneracional de variadas
actitudes, creencias y comportamientos políticos de estudiantes y
jóvenes, sin embargo hasta ahora se existen escasos estudios que
consideran opiniones y creencias de los jóvenes sobre la justicia en la
sociedad. ¿Creen los jóvenes que la desigualdad actual es justa o
injusta? Más específicamente, ¿en qué medida consideran los jóvenes que
los recursos se distribuyen meritocráticamente según el esfuerzo? Estas
son algunas preguntas del campo de la justicia distributiva que tendría
sentido realizar en el marco de la socialización política, ya que
creencias meritocraticas han sido vinculadas a temas cívicos como el
autoritarismo social, la justificación de las desigualdades y la
intolerancia (Azevedo et al.,
\protect\hyperlink{ref-azevedo_Neoliberal_2019}{2019}; Bay-Cheng et al.,
\protect\hyperlink{ref-bay-cheng_Tracking_2015a}{2015}; Madeira et al.,
\protect\hyperlink{ref-madeira_Primes_2019}{2019}). En vista de lo
anterior la pregunta general de este estudio es: ¿En qué medida la
familia y la escuela afectan la percepción de la meritocracia de los
estudiantes? La percepción de meritocracia se refiere a una constatación
u observación personal sobre el funcionamiento de la meritocracia en la
sociedad, entendida como que los recursos se distribuyen en base a
mérito (esfuerzo y talento), en lugar de otros factores como origen
social o contactos personales. Este estudio busca analizar las
percepciones de estudiantes de 2do medio en Chile (n = 1635) en relación
con dos agentes de socialización: la familia y la escuela. Se hipotetiza
en términos generales que las creencias meritocráticas de los
estudiantes son influidas por características de la familia, como las
opiniones y los recursos, así como también por vivencias en la escuela,
como la experiencia de justicia en las calificaciones obtenidas.

\pagebreak

\hypertarget{se-han-recopilado-ya-datos-para-este-estudio}{%
\subsection{¿Se han recopilado ya datos para este
estudio?}\label{se-han-recopilado-ya-datos-para-este-estudio}}

Los datos provienen de un estudio sobre socialización política familiar.
Corresponden a una muestra representativa de los estudiantes de segundo
medio (décimo grado) que asisten a escuelas de 3 regiones chilenas:
Región de Antofagasta, Región Metropolitana y Región del Maule. La
muestra es representativa de los distintos tipos de establecimientos
educacionales existentes en el país (Particulares pagados, Particulares
Subvencionados y Municipales o públicos). En total se seleccionaron 64
escuelas (14 establecimientos de Antofagasta, 35 de la Metropolitana y
15 del Maule). En cada establecimiento educacional seleccionado se
consideró encuestar a un curso completo, sus apoderados y sus docentes
de historia, ciencias sociales y/o formación ciudadana. En concreto, se
cuenta con los datos de 1635 estudiantes, 744 apoderados y 103
profesores. Los datos fueron producidos entre agosto y diciembre del año
2019.

\hypertarget{cuuxe1l-es-la-pregunta-principal-o-la-hipuxf3tesis-que-se-estuxe1-probando-en-este-estudio}{%
\subsection{¿Cuál es la pregunta principal o la hipótesis que se está
probando en este
estudio?}\label{cuuxe1l-es-la-pregunta-principal-o-la-hipuxf3tesis-que-se-estuxe1-probando-en-este-estudio}}

\textbf{¿En que medida la familia y la escuela afectan la percepción de
la meritocracia de los estudiantes?} La percepción de meritocracia se
refiere a una constatación u observación personal sobre el
funcionamiento de la meritocracia en la sociedad, entendida como la
percepción de que los recursos se distribuyen en base a mérito (esfuerzo
y talento). En el caso específico de este estudio el interés principal
es analizar las percepciones de estudiantes de 2do medio en Chile en
relación a dos agentes de socialización: la familia y la escuela. Se
señala en términos generales que las creencias meritocráticas de los
estudiantes son influidas por los recursos y opiniones de la familia así
como también por la experiencia de justicia o injusticia en relación a
las calificaciones obtenidas en la escuela.

\hypertarget{hipuxf3tesis}{%
\subsubsection{Hipótesis}\label{hipuxf3tesis}}

\emph{Hipótesis principales de socialización de la meritocracia}

\begin{itemize}
\item
  \(H_1\): Estudiantes cuyos padres manifiestan una mayor percepción de
  meritocracia, perciben mayor meritocracia.
\item
  \(H_2\): Estudiantes con una mayor sensación de justicia en sus notas
  percibirán mayor meritocracia.
\end{itemize}

\emph{Hipótesis de moderación}

\begin{itemize}
\tightlist
\item
  \(H_3\): La relación entre percepción meritocrática de los padres y de
  los hijos (H1) será más positiva para aquellos que experimentan un
  mayor sentido de justicia en la escuela.
\end{itemize}

\emph{Hipótesis de contexto socioeconómico}

\begin{itemize}
\item
  \(H_4\): El estatus socioeconómico de la familia posee un efecto
  positivo sobre las percepciones meritocráticas del estudiante
\item
  \(H_5\): El estatus socioeconómico de la escuela posee un efecto
  positivo sobre las percepciones meritocráticas del estudiante
\end{itemize}

\emph{Hipótesis de mediación}

\begin{itemize}
\item
  \(H_6\): La percepción meritocrática de los padres media la relación
  entre el estatus socioeconómico de los padres y la percepción
  meritocrática de los estudiantes.
\item
  \(H_7\): La sensación de justicia en las notas media la relación entre
  el estatus socioeconómico de la escuela y la percepción meritocrática
  estudiantes.
\end{itemize}

\hypertarget{describa-las-variables-claves-especificando-cuxf3mo-se-mediruxe1n.}{%
\subsection{Describa la(s) variable(s) clave(s) especificando cómo se
medirán.}\label{describa-las-variables-claves-especificando-cuxf3mo-se-mediruxe1n.}}

Las variables más relevantes para el estudio se muestran a continuación:

\begin{table}[!h]

\caption{\label{tab:table-dependientes}Variables dependientes.}
\centering
\fontsize{10}{12}\selectfont
\begin{tabu} to \linewidth {>{\raggedright\arraybackslash}p{2 cm}>{\raggedright\arraybackslash}p{7 cm}>{\raggedright\arraybackslash}p{4 cm}}
\toprule
Variable & Pregunta & Categorías de respuesta\\
\midrule
 &  & (1) Muy en desacuerdo\\
\cmidrule{3-3}
 & \multirow{-2}{7 cm}{\raggedright\arraybackslash En Chile, los que se esfuerzan salen adelante} & (4) Muy de acuerdo\\
\cmidrule{2-3}
 &  & (1) Nada importante\\
\cmidrule{3-3}
\multirow{-4}{2 cm}{\raggedright\arraybackslash Percepción Meritocrática} & \multirow{-2}{7 cm}{\raggedright\arraybackslash Actualmente en Chile, para surgir en la vida ¿Cuán importante es: El trabajo duro?} & (4) Muy importante\\
\bottomrule
\end{tabu}
\end{table}

La variable dependiente de este estudio es la percepción meritocrática
de los estudiantes. Esta se medirá a partir de dos indicadores, uno
respecto a la opinión del éxito y el esfuerzo, y otro respecto a la
importancia del trabajo duro. Ambos indicadores corresponden a escalas
Likert de cuatro categorías, midiendo grado de acuerdo y grado de
importancia, respectivamente.

\begin{table}[!h]

\caption{\label{tab:table-independientesn1}Variables independientes de nivel 1.}
\centering
\fontsize{10}{12}\selectfont
\begin{tabu} to \linewidth {>{\raggedright\arraybackslash}p{3 cm}>{\raggedright\arraybackslash}p{7 cm}>{\raggedright\arraybackslash}p{4 cm}}
\toprule
Variable & Pregunta & Categorías de respuesta\\
\midrule
 &  & (1) Muy en desacuerdo\\
\cmidrule{3-3}
 & \multirow{-2}{7 cm}{\raggedright\arraybackslash En Chile, los que se esfuerzan salen adelante} & (4) Muy deacuerdo\\
\cmidrule{2-3}
 &  & (1) Nada  importante\\
\cmidrule{3-3}
\multirow{-4}{3 cm}{\raggedright\arraybackslash Percepción Meritocrática de los padres} & \multirow{-2}{7 cm}{\raggedright\arraybackslash Actualmente en Chile, para surgir en la vida ¿Cuán importante es: El trabajo duro?} & (4) Muy importante\\
\cmidrule{1-3}
 &  & (min) 1\\
\cmidrule{3-3}
 & \multirow{-2}{7 cm}{\raggedright\arraybackslash ¿Qué promedio de notas obtuviste el año pasado?} & (max) 7\\
\cmidrule{2-3}
 &  & (min) 1\\
\cmidrule{3-3}
\multirow{-4}{3 cm}{\raggedright\arraybackslash Sentido de justicia en la escuela (Indirecto)} & \multirow{-2}{7 cm}{\raggedright\arraybackslash ¿Qué promedio de nota piensas que merecías?} & (max) 7\\
\cmidrule{1-3}
 &  & (1) Menos de las que merezco\\
\cmidrule{3-3}
 &  & (2)Las que merezco\\
\cmidrule{3-3}
\multirow{-3}{3 cm}{\raggedright\arraybackslash Sentido de justicia en la escuela (Directo)} & \multirow{-3}{7 cm}{\raggedright\arraybackslash Tomando en cuenta el tiempo que le dedico a mis estudios, las notas que me saco son…} & (3)Más de las que merezco\\
\cmidrule{1-3}
 &  & (1) Menos de \$101.000 mensuales liquidos\\
\cmidrule{3-3}
 & \multirow{-2}{7 cm}{\raggedright\arraybackslash A continuación, le presentamos un listado de rangos de ingreso. ¿Podría usted indicarme en cuál de ellos se encuentra el ingreso total de su hogar en el último mes?} & (11) Más de \$3.000.000 mensuales liquidos\\
\cmidrule{2-3}
 &  & (1) 8vo básico o menos\\
\cmidrule{3-3}
\multirow{-4}{3 cm}{\raggedright\arraybackslash Estatus socioeconómico de los padres} & \multirow{-2}{7 cm}{\raggedright\arraybackslash ¿Cuál es el último curso o nivel de estudios que completó usted?} & (5) Una carrera en la Universidad o estudios de posgrado\\
\bottomrule
\end{tabu}
\end{table}

En cuanto a las variables independientes, a nivel individual pueden
dividirse de acuerdo a si corresponden a padres o estudiantes. En el
caso de los padres, existen dos variables relevantes: percepción de
meritocracia y estatus socioeconómico. Para la percepción de
meritocracia los indicadores y categorías son idénticas a la variable
dependiente utilizada en el cuestionario de estudiantes. Para el estatus
socioeconómico, el indicador a utilizar serán los ingresos declarados
por los padres, el cuál cuenta con 11 tramos de ingresos. En el caso de
los estudiantes, se utiliza una pregunta para medir directamente la
sensación de justicia en las notas y otro que lo hace de manera
indirecta. El indicador de medición indirecta de justicia en las notas
se obtiene mediante una proporción entre el reporte de la nota promedio
obtenida el año anterior (recompensa obtenida) y el reporte sobre la
nota considerada justa. A partir de estos indicadores se utilizará la
formula de Jasso (\protect\hyperlink{ref-jasso_New_1980}{1980}) sobre la
evaluación de justicia, siguiendo las aplicaciones al ámbito educativo
por parte de Resh \& Sabbagh
(\protect\hyperlink{ref-resh_Sense_2014}{2014}), Resh \& Sabbagh
(\protect\hyperlink{ref-resh_Sense_2017}{2017}), Resh
(\protect\hyperlink{ref-resh_Sense_2018}{2018}). La formula corresponde
al logaritmo natural de la proporción entre la recompensa obtenida y la
recompensa justa, en este caso las notas:

\(\text{Sentido de justicia en notas}= ln(\frac{\text{nota obtenida}}{\text{nota justa}})\)

\pagebreak

\begin{table}[!h]

\caption{\label{tab:table-independientesn2}Variables independientes de nivel 2.}
\centering
\fontsize{10}{12}\selectfont
\begin{tabu} to \linewidth {>{\raggedright\arraybackslash}p{4 cm}>{\raggedright\arraybackslash}p{6 cm}>{\raggedright\arraybackslash}p{4 cm}}
\toprule
Variable & Pregunta & Categorías de respuesta\\
\midrule
 &  & (min) \$212.250\\
\cmidrule{3-3}
 & \multirow{-2}{6 cm}{\raggedright\arraybackslash Promedio de los ingresos de las familias dentro de la escuela} & (max) \$1.450.929\\
\cmidrule{2-3}
 &  & (min) 0,00\%\\
\cmidrule{3-3}
\multirow{-4}{4 cm}{\raggedright\arraybackslash Estatus socioeconómico de la escuela} & \multirow{-2}{6 cm}{\raggedright\arraybackslash Proporción de padres con educación universitaria} & (max) 83,3\%\\
\bottomrule
\end{tabu}
\end{table}

Para calcular el estatus socioeconómico de la escuela se utilizará el
promedio de estatus socioeconómico de los padres. Se utilizara el valor
intermedio de cada rango de ingresos para calcular el promedio de la
escuela, utilizando el estatus socioeconómico familiar como una variable
cuantitativa. El nivel educativo de los padres a nivel escuela se
trabajara como una variable cuantitativa generada a partir de la
proporción de padres con educación universitaria.

Las variables control incluirán edad, sexo, posición política de padres
y estudiantes, ciudad, dependencia administrativa de la escuela,
heterogeneidad socioeconómica de la escuela y cantidad de libros en el
hogar.

\hypertarget{especifique-exactamente-quuxe9-anuxe1lisis-se-realizaruxe1-para-examinar-la-pregunta-hipuxf3tesis-principal}{%
\subsection{Especifique exactamente qué análisis se realizará para
examinar la pregunta / hipótesis
principal}\label{especifique-exactamente-quuxe9-anuxe1lisis-se-realizaruxe1-para-examinar-la-pregunta-hipuxf3tesis-principal}}

Debido a que la muestra posee una estructura jerárquica (estudiantes
anidados en escuelas), se estimarán regresiones multinivel para
contrastar las hipótesis siguiendo los pasos recomendados para este tipo
de modelos (Aguinis et al., 2013). El estudio considera 1635 estudiantes
participantes anidados en 64 escuelas.

Esta investigación posee dos variables dependientes de nivel ordinal. En
un primer momento se analizará la distribución de ellas así como su
grado de correlación, y en caso que su correlación sea estadísticamente
significativa a ambos niveles y con un tamaño de efecto al menos
moderado, se construirá un índice promedio simple que será la variable
dependiente del estudio. En este caso se utilizará el estimador de
máxima verosimilitud restringida para modelos con efectos aleatorios
mediante la librería \texttt{lme4} de \texttt{R}. En caso que no se
fundamente apropiadamente la construcción de un índice, se trabajará con
modelos multinivel para variables ordinales (Arfan \& Sherwani,
\protect\hyperlink{ref-arfan_Ordinal_2017}{2017})\footnote{En modelos
  ordinales multinivel la variables dependiente representa el cálculo de
  la probabilidad acumulada de que un estudiante responda hasta \(C\) en
  los indicadores de percepción de meritocracia, siendo \(Y_{cij}\) una
  respuesta categórica ordenada de un estudiante \(i^{th}\) en una
  escuela (cluster) \(j^{th}\) con \(C\) categorías ordenadas,
  codificadas como \(C = 1,2,3,4\). Esta probabilidad se calcula en
  función de: el intercepto por cada categoría \(a_c\), los coeficientes
  \(\gamma_1\) y \(\gamma_2\) para la percepción de meritocracia de los
  padres y el sentido de justicia en las notas respectivamente,
  \(\gamma_n\) para las variables de control, \(u_{0j}\) como termino de
  error para una escuela \(j\) y \(e_{ij}\) como error de la estimación
  para el individuo \(i\) en una escuela \(j\). Para la estimación se
  utilizarán librerías especializadas de R como \texttt{clmm2} y
  \texttt{mvord} (Hirk et al.,
  \protect\hyperlink{ref-hirk_mvord_2020}{2020})} .

Los modelos serán estimados incorporando las variables independientes
asociadas a cada hipótesis más las variables de control.

El análisis se llevará a cabo con el software estadístico \texttt{R}
versión 4.0.3 ``Bunny-Wunnies Freak Out''.

\hypertarget{alguxfan-anuxe1lisis-secundario}{%
\subsection{¿Algún análisis
secundario?}\label{alguxfan-anuxe1lisis-secundario}}

Como análisis de robustez de los modelos multinivel se realizará la
prueba D-cook para detectar casos influyentes, contrastando
posteriormente la estimación con y sin ellos. En la misma línea se
calcularán los dfbetas para cada predictor. Con el mismo objetivo,
compararemos los ajustes de los modelos multinivel con los mismos
modelos pero con las variables centradas al promedio del grupo, para
evaluar que el efecto que señalamos como individual no se deba al
contexto de la escuela.

\hypertarget{cuuxe1ntas-observaciones-se-recopilaruxe1n-o-que-determinaruxe1-el-tamauxf1o-de-la-muestra-no-es-necesario-justificar-la-decisiuxf3n-pero-sea-preciso-sobre-cuxf3mo-se-determinaruxe1-exactamente-el-nuxfamero.}{%
\subsection{¿Cuántas observaciones se recopilarán o que determinará el
tamaño de la muestra? No es necesario justificar la decisión, pero sea
preciso sobre cómo se determinará exactamente el
número.}\label{cuuxe1ntas-observaciones-se-recopilaruxe1n-o-que-determinaruxe1-el-tamauxf1o-de-la-muestra-no-es-necesario-justificar-la-decisiuxf3n-pero-sea-preciso-sobre-cuxf3mo-se-determinaruxe1-exactamente-el-nuxfamero.}}

Se utilizan datos secundarios descritos anteriormente.

\hypertarget{algo-muxe1s-que-le-gustaruxeda-agregar-por-ejemplo-exclusiones-de-datos-variables-recopiladas-con-fines-exploratorios-anuxe1lisis-inusuales-previstos}{%
\subsection{¿Algo más que le gustaría agregar? (por ejemplo, exclusiones
de datos, variables recopiladas con fines exploratorios, análisis
inusuales
previstos)}\label{algo-muxe1s-que-le-gustaruxeda-agregar-por-ejemplo-exclusiones-de-datos-variables-recopiladas-con-fines-exploratorios-anuxe1lisis-inusuales-previstos}}

Para las regresiones multinivel, en miras del tamaño de la muestra se
consideran significativas las relaciones con un \(p< 0,05\). Para
calcular el R2 de las relaciones se utilizara la técnicas de Bryk \&
Raudenbush (\protect\hyperlink{ref-bryk_Hierarchical_1992}{1992}).

\pagebreak

\hypertarget{referencias}{%
\subsection*{Referencias}\label{referencias}}
\addcontentsline{toc}{subsection}{Referencias}

\hypertarget{refs}{}
\leavevmode\hypertarget{ref-arfan_Ordinal_2017}{}%
Arfan, M., \& Sherwani, R. A. K. (2017). Ordinal Logit and Multilevel
Ordinal Logit Models: An Application on Wealth Index MICS-Survey Data.
\emph{Pakistan Journal of Statistics and Operation Research}, 211--226.
\url{https://doi.org/10.18187/pjsor.v13i1.1801}

\leavevmode\hypertarget{ref-azevedo_Neoliberal_2019}{}%
Azevedo, F., Jost, J. T., Rothmund, T., \& Sterling, J. (2019).
Neoliberal Ideology and the Justification of Inequality in Capitalist
Societies: Why Social and Economic Dimensions of Ideology Are
Intertwined: Neoliberal Ideology and Justification. \emph{Journal of
Social Issues}, \emph{75}(1), 49--88.
\url{https://doi.org/10.1111/josi.12310}

\leavevmode\hypertarget{ref-bay-cheng_Tracking_2015a}{}%
Bay-Cheng, L. Y., Fitz, C. C., Alizaga, N. M., \& Zucker, A. N. (2015).
Tracking Homo Oeconomicus: Development of the Neoliberal Beliefs
Inventory. \emph{Journal of Social and Political Psychology},
\emph{3}(1), 71--88. \url{https://doi.org/10.5964/jspp.v3i1.366}

\leavevmode\hypertarget{ref-bryk_Hierarchical_1992}{}%
Bryk, A. S., \& Raudenbush, S. W. (1992). \emph{Hierarchical linear
models: Applications and data analysis methods} (pp. xvi, 265). Sage
Publications, Inc.

\leavevmode\hypertarget{ref-hirk_mvord_2020}{}%
Hirk, R., Hornik, K., \& Vana, L. (2020). Mvord: An R Package for
Fitting Multivariate Ordinal Regression Models. \emph{Journal of
Statistical Software}, \emph{93}(1), 1--41.
\url{https://doi.org/10.18637/jss.v093.i04}

\leavevmode\hypertarget{ref-jasso_New_1980}{}%
Jasso, G. (1980). A New Theory of Distributive Justice. \emph{American
Sociological Review}, \emph{45}(1), 3--32.
\url{https://doi.org/10.2307/2095239}

\leavevmode\hypertarget{ref-madeira_Primes_2019}{}%
Madeira, A. F., Costa-Lopes, R., Dovidio, J. F., Freitas, G., \&
Mascarenhas, M. F. (2019). Primes and Consequences: A Systematic Review
of Meritocracy in Intergroup Relations. \emph{Frontiers in Psychology},
\emph{10}. \url{https://doi.org/10.3389/fpsyg.2019.02007}

\leavevmode\hypertarget{ref-resh_Sense_2018}{}%
Resh, N. (2018). Sense of Justice in School and Social and Institutional
Trust. \emph{Comparative Sociology}, \emph{17}(3-4), 369--385.
\url{https://doi.org/10.1163/15691330-12341465}

\leavevmode\hypertarget{ref-resh_Sense_2014}{}%
Resh, N., \& Sabbagh, C. (2014). Sense of justice in school and civic
attitudes. \emph{Social Psychology of Education}, \emph{17}(1), 51--72.
\url{https://doi.org/10.1007/s11218-013-9240-8}

\leavevmode\hypertarget{ref-resh_Sense_2017}{}%
Resh, N., \& Sabbagh, C. (2017). Sense of justice in school and civic
behavior. \emph{Social Psychology of Education}, \emph{20}(2), 387--409.
\url{https://doi.org/10.1007/s11218-017-9375-0}

\end{document}
